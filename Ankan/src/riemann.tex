\begin{frame}
\frametitle{Riemann Curvature from Sectional Curvature}
\begin{block}{Riemann Curvature from Sectional Curvature}
We can deduce the Riemann curvature at a point in the manifold given the sectional curvature of the surface.
\end{block}
\end{frame}


\begin{frame}
\frametitle{Motivation}
We will define some basic assumptions that we will use in out proof.
\begin{itemize}
\item Define 
\begin{align}
R(x,y,z,w)=\inner{R(x,y)z,w}
\end{align} 
\item Skew symmetry
\begin{align}
\label{skew	1}
R(x,y,z,w)&=-R(y,x,z,w)\\
\label{skew	2}
R(x,y,z,w)&=-R(x,y,w,z)
\end{align} 
\item Symmetry
\begin{align}
\label{sym}
R(x,y,z,w)=R(z,w,x,y)
\end{align}
\item Bianchi Identity
\begin{align}
\label{bianchi}
\sum_{\pi(x,y,z)}R(x,y,z,w)=0
\end{align}
\item Sectional Curvature
\begin{align}
K(v,w)=\bfrac{R(v,w,w,v)}{\norm{v\wedge w}^2}
\end{align}
\end{itemize}
\end{frame}

\begin{frame}
\frametitle{Setting Up}
Now, define a polynomial
\begin{align}
\label{polynomial f}
f(t)=&R(x+tw,y+tz,y+tz,x+tw)\\
&-t^2(R(x,z,z,x)+R(w,y,y,w))
\end{align}
Since all terms are symmetric, $f(t)$ can be expressed in terms of the sectional curvature and norm of wedge products. Thus, the RHS of equation \ref{polynomial f} is a polynomial in $t$ whose coefficients are the sectional curvatures.\\ \pause
Consider the coefficient of $t^2$ in equation \ref{polynomial f}. Using the multi linearity property of $R$ tensor, we can write it as
\begin{align}
f''(0)=R(x,y,z,w)+R(x,z,y,w)+R(w,z,y,x)+R(w,y,z,x)
\end{align}
\end{frame}

\begin{frame}
Using equations \ref{skew	1}, \ref{skew	2}, \ref{sym}, we obtain
\begin{align}
\label{coeff in f}
f''(0)=2R(x,y,z,w)-2R(z,x,y,w)
\end{align}
\frametitle{Some Simple Manipulations}
Now, exchange $x$ and $y$ and define a new polynomial
\begin{align}
\label{polynomial g}
g(t)=&R(y+tw,x+tz,x+tz,y+tw)\\
&-t^2(R(y,z,z,y)+R(w,x,x,w))
\end{align}
Since all terms are symmetric, $g(t)$ can again be expressed in terms of the sectional curvature and norm of wedge products. Following a similar procedure, we obtain
\begin{align}
\label{coeff in g}
g''(0)=-2R(x,y,z,w)+2R(y,z,x,w)
\end{align}
\pause
Now, \ref{coeff in f}-\ref{coeff in g} gives
\begin{align}
\label{riemann xyz}
f''(0)-g''(0)=4R(x,y,z,w)-2R(z,x,y,w)-2R(y,z,x,w)
\end{align}
\end{frame}

\begin{frame}
\frametitle{Completion of Proof}
Using \ref{bianchi},
\begin{align}
\label{substitution}
R(z,x,y,w)+R(y,z,x,w)=-R(x,y,z,w)
\end{align}
Substituting \ref{substitution} in \ref{riemann xyz},
\begin{align}
\label{final form}
f''(0)-g''(0)&=6R(x,y,z,w)\\
\Rightarrow &R(x,y,z,w)=\bfrac{f''(0)-g''(0)}{6}=\bfrac{(f-g)''(0)}{6}
\end{align}
\end{frame}

\begin{frame}
\frametitle{Final Result}
We need the polynomials $f$ and $g$ in terms of sectional curvature. Observe the form of $f$ in \ref{polynomial f}. Using the formula for sectional curvature,
\begin{align}
\label{f as sectional}
f(t)=&K(x+tw,y+tz)\norm{(x+tw)\wedge(y+tz)}^2\\
&-t^2(K(x,z)\norm{x\wedge z}^2+K(w,y)\norm{w\wedge y}^2)
\end{align}
Likewise
\begin{align}
\label{g as sectional}
g(t)=&K(y+tw,x+tz)\norm{(y+tw)\wedge(x+tz)}^2\\
&-t^2(K(y,z)\norm{y\wedge z}^2+K(w,x)\norm{w\wedge x}^2)
\end{align}
\end{frame}

\begin{frame}
\frametitle{Halloween Special: A Scary Expression}
Complete evaluation of equations \ref{final form}, \ref{f as sectional}, and \ref{g as sectional} gives an explicit formula for $R(x,y,z,w)$.
\pause
{\footnotesize
\begin{align*}
6\inner{R(x,y)z,w}=&K(x+w,y+z)\norm{(x+w)\wedge (y+z)}^2\\
&-K(y+w,x+z)\norm{(y+w)\wedge (x+z)}^2\\
&-K(x,y+z)\norm{x\wedge (y+z)}^2-K(y,x+w)\norm{y\wedge (x+w)}^2\\
&-K(z,x+w)\norm{z\wedge (x+w)}^2-K(w,y+z)\norm{w\wedge (y+z)}^2\\
&+K(x,y+w)\norm{x\wedge (y+w)}^2+K(y,z+w)\norm{y\wedge (z+w)}^2\\
&+K(z,y+w)\norm{z\wedge (y+w)}^2+K(w,x+z)\norm{w\wedge (x+z)}^2\\
&+K(x,z)\norm{x\wedge z}^2+K(y,w)\norm{y\wedge w}^2\\ 
&-K(x,w)\norm{x\wedge w}^2-K(y,z)\norm{y\wedge z}^2\\ 
\end{align*}
}%
Quite scary, isn't it?
\end{frame}
